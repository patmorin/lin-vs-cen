\documentclass{patmorin}
\listfiles
\usepackage{pat}
\usepackage{paralist}
\usepackage{dsfont}  % for \mathds{A}
\usepackage[utf8x]{inputenc}
\usepackage{skull}
\usepackage{paralist}
\usepackage{graphicx}
\usepackage[noend]{algorithmic}

\usepackage[normalem]{ulem}
\usepackage{cancel}
%\usepackage{enumitem}

\usepackage{todonotes}

% etoolbox allows for robust commands that don't need \protect, e.g.
% \newrobustcmd{\onesub}{\mathord{\includegraphics{figs/one-sub}}}
% \subsection{Approximate Voronoi Diagrams in $G^{\onesub}_k$}
\usepackage{etoolbox}

\usepackage[longnamesfirst,numbers,sort&compress]{natbib}

\usepackage[mathlines]{lineno}
\setlength{\linenumbersep}{2em}
% \linenumbers
% \rightlinenumbers
% \linenumbers
\newcommand*\patchAmsMathEnvironmentForLineno[1]{%
 \expandafter\let\csname old#1\expandafter\endcsname\csname #1\endcsname
 \expandafter\let\csname oldend#1\expandafter\endcsname\csname end#1\endcsname
 \renewenvironment{#1}%
    {\linenomath\csname old#1\endcsname}%
    {\csname oldend#1\endcsname\endlinenomath}}%
\newcommand*\patchBothAmsMathEnvironmentsForLineno[1]{%
 \patchAmsMathEnvironmentForLineno{#1}%
 \patchAmsMathEnvironmentForLineno{#1*}}%
\AtBeginDocument{%
\patchBothAmsMathEnvironmentsForLineno{equation}%
\patchBothAmsMathEnvironmentsForLineno{align}%
\patchBothAmsMathEnvironmentsForLineno{flalign}%
\patchBothAmsMathEnvironmentsForLineno{alignat}%
\patchBothAmsMathEnvironmentsForLineno{gather}%
\patchBothAmsMathEnvironmentsForLineno{multline}%
}


% Taken from
% https://tex.stackexchange.com/questions/42726/align-but-show-one-equation-number-at-the-end
\newcommand\numberthis{\addtocounter{equation}{1}\tag{\theequation}}


\setlength{\parskip}{1ex}

% Document-specific commands and math operators
\DeclareMathOperator{\tw}{tw}
\DeclareMathOperator{\chicen}{\chi_{\mathrm{cen}}}
\DeclareMathOperator{\chilin}{\chi_{\mathrm{lin}}}
\DeclareMathOperator{\dist}{dist}
\DeclareMathOperator{\vor}{Vor}

\newrobustcmd{\onesub}{\mathord{\includegraphics{figs/one-sub}}}
\newrobustcmd{\leftup}{\mathord{\includegraphics{figs/left-up}}}

\title{\MakeUppercase{Linear versus centred Colouring Numbers}\thanks{This research was partly funded by NSERC.}}
\author{Prosenjit Bose, Vida Dujmović, Mehrnoosh Javarsineh, and Pat Morin}

\date{}


\begin{document}

\maketitle

\begin{abstract}
    These are some notes on the relationships betwee linear colouring numbers and centred colouring numbers.
\end{abstract}

\section{Introduction}

Let $G$ be a simple undirected graph.  A \emph{$k$-colouring} of $G$ is any function $\varphi:V(G)\to S$ where $S$ is a set of size $k$.  A vertex $v$ of $G$ is a \emph{centre} of $G$ with respect to $\varphi$ if $\varphi(v)\not\in\{\varphi(w):w\in V(G)\setminus\{v\}\}$, i.e., $v$ is the unique vertex of $G$ having colour $\varphi(v)$.  A colouring $\varphi$ of $G$ is \emph{centred} if every connected subgraph of $G$ has a centre with respect to $\varphi$. A colouring $\varphi$ of $G$ is \emph{linear} if every simple path in $G$ has a centre with respect to $\varphi$.

The \emph{centred chromatic number} $\chicen(G)$ of $G$ is the minimum integer $c$ such that $G$ has a centred $c$-colouring.  The \emph{linear chromatic number} of $G$ is the minimum integer $\ell$ such that $G$ has a linear $\ell$-colouring.  Since every path in $G$ is a connected subgraph of $G$, every centred colouring of $G$ is also a linear colouring of $G$, so $\chilin(G)\le\chicen(G)$.

The question of upper bounding $\chicen(G)$ by some function of $\chilin(G)$ was considered by \citet[Theorem~1]{kun.obrien.ea:polynomial} who showed the following result:

\begin{thm}[\citet{kun.obrien.ea:polynomial}]\label{kun_obrien_general}
  For any graph $G$, $\chicen(G)\le \chilin^{190}(G)\cdot\log^{O(1)}(\chilin(G))$.
\end{thm}


Their proof of \cref{kun_obrien_general} has three steps:
\begin{enumerate}
  \item A theorem of \citet{kawarabayashi.rossman:polynomial} shows that, if $\chicen(G)\ge k^{190}\log^{O(1)} k$ then $\chilin(G)\ge k^{38}$ or $G$ has treewidth $\tw(G)\ge k^{38}$.  In the former case there is nothing left to prove.
  \item The current-best version of the Excluded Grid Theorem due to \citet{chuzhoy:improved} shows that, if $\tw(G)\ge k^{38}$, then $G$ contains an $\Omega(k^2\times k^2)$ grid minor.
  \item A technical lemma \cite[Lemma~5]{kun.obrien.ea:polynomial} shows that, if $G$ contains a $k^2\times k^2$ grid minor, then $\chilin(G)\in\Omega(k)$.
\end{enumerate}

These notes are an attempt to improve the bound in \cref{kun_obrien_general} in the general case as well as for some special classes of graphs.

As a first step, we attempt to improve the third part of the argument to show that, if $G$ contains a $k\times k$ grid minor, then $\chilin(G)\in \Omega(k)$.  This, by itself, reduces the exponent in \cref{kun_obrien_general} from $190$ to $95$.  Next, we observe that the first two parts of the argument are both ``heavy hammers'' and that, for some special graph classes, much lighter hammers suffice.  For example, if we consider only planar graphs, then it is well known that any planar graph $G$ of treewidth $k$ contains an $\Omega(k\times k)$ grid minor.  This already reduces the exponent further (for planar graphs) to $5$.  If one could prove a similar result for the first ``heavy hammer'' then this would show that, for any planar graph $G$, $\chicen(G)\in O(\chilin(G))$.


\section{Preliminaries}

For a graph $G$ and a vertex $v\in V(G)$, $\deg_G(v):=|\{vw\in E(G)\}|$ denotes the degree of $v$ in $G$.  For a any $S\subseteq V(G)$, $\deg_G(S):=\sum_{v\in S}\deg_G(v)$ is the total degree of $S$.

For a vertex $v$ in a graph $G$, the \emph{open neighbourhood} of $v$ is $N_G(v):=\{w:vw\in E(G)\}$ and the \emph{closed neighbourhood} of $v$ is $N_G[v]:=\{v\}\cup N_G(v)$.  For any $S\subseteq V(G)$, $N_G(s):=\bigcup_{v\in S}N_G(v)$ and $N_G[s]:=\bigcup_{v\in S}N_G[v]$.  The subgraph $G[S]$ of $G$ \emph{induced} by $S$ is the graph with vertex set $V(G[S]):=S\cap V(G)$ and edge set $E(G[S]):=\{vw\in E(G):\{v,w\}\subseteq S\}$.


For a graph $G$ and a partition $\mathcal{P}$ of $V(G)$, the \emph{quotient graph} $G/\mathcal{P}$ is the graph with vertex set $\mathcal{P}$ and that contains an edge $vw$ if and only $G$ contains an edge with one endpoint in $v$ and one endpoint in $w$.  When $\mathcal{P}$ consists of disjoint subsets of $V(G)$ that do not necessarily cover $V(G)$ then we use the convention that $G/\mathcal{P}:=G/\mathcal{P'}$ where $\mathcal{P}'$ is the completion of $\mathcal{P}$ obtained by introducing each missing element of $V(G)$ as singleton set.  More precisely $\mathcal{P'}:=\mathcal{P}\cup\{\{v\}:v\in V(G)\setminus(\cup P)\}$.


A \emph{subdivision} of a graph $G$ is any graph $G'$ that can be obtained from $G$ by replacing each edge $vw$ of $G$ with a path $v,s_1,s_2,\ldots,s_\ell,w$ in which each of the (newly introduced) internal vertices $s_1,\ldots,s_\ell$ have degree exactly $2$.  Each vertex in $V(G')\setminus V(G)$ is called a \emph{subdivision vertex}.  A $\ell$-subdivision of $G$ is a subdivision of $G$ in which each edge of $G$ is replaced by a path that has exactly $\ell$ internal vertices.  For any graph $G$, we use $G^{\onesub}$ to denote the $1$-subdivision of $G$.  Note that there is a natural bijection between the vertex set $V(G^{\onesub})$ and the vertex/edge set $V(G)\cup E(G)$.  A $\le\!\!\ell$-subdivision of $G$ is a subdivision of $G$ in which each edge of $G$ is replaced by a path that has at most $\ell$ internal vertices.


\todo[inline]{Surely this next lemma is well-known?}
We make use of the following generalization of Hall's Marriage Theorem \cite{hall:on}:
\begin{lem}\label{hall_vees}
  Let $G$ be a bipartite graph with vertex parts $A$ and $B$ and having the property that, for any $A_0\subseteq A$, $|N_G(W)|\ge 2|A_0|$.  Then $G$ contains a subgraph $\Lambda$ in which each vertex of $A$ has degree exactly two and each vertex of $B$ has degree at most one.
\end{lem}

\begin{proof}
  Let $\Lambda$ be a subgraph of $G$ in which each vertex of $A$ has degree at most two, each vertex of $B$ has degree at most one, and that maximizes $\deg_\Lambda(A)$.  If $\deg_{\Lambda}(A) = 2|\Lambda|$ then there is nothing to prove. Assume therefore, for the sake of contradiction,  that there exists $v_0\in A$ with $\deg_\Lambda(v_0) < 2$.

  We say that a path $v_0,\ldots,v_m$ in $G$ is \emph{$\Lambda$-aternating} if $v_iv_{i+1}\in E(\Lambda)$ for all odd $i\in\{0,\ldots,m-1\}$ and $v_iv_{i+1}\not\in E(\Lambda)$ for all even $i\in\{0,\ldots,m\}$.  Observe that any such path has even length so that it ends at a vertex $v_m\in A$.  Otherwise, replacing the edges $\lfloor m/2\rfloor$ edges $\{v_1v_2,v_3v_4,\ldots,v_{m-2}v_{m-1}\}\in E(\Lambda)$ with the $\lceil m/2\rceil$ edges $\{v_0v_1,v_2v_3,\ldots,v_{m-1}v_m\}$ does not change $\deg_\Lambda(v)$ for any $v\in A_0\setminus\{v_0\}$ but does increase  $\deg_\Lambda(v_0)$, contradicting the assumption that $\Lambda$ maximizes $\deg_\Lambda(A)$.

  Let $A_0\subseteq A$ and $B_0\subseteq B$ be the subsets of $A$ and $B$, respectively, that can be reached from $v_0$ by $\Lambda$-alternating paths.
  Observe that, for any $x\in B_0$,  there is exactly one edge $vx$ of $\Lambda$ incident on $x$ and the other endpoint $v$ of this edge is in $A_0$.  Let $a$ be the number of edges of $\Lambda$ having one endpoint $A_0$ and one endpoint in $B_0$, so that $a=\deg_\Lambda(B_0)=|B_0|$.  Let $b$ be the number of edges of $\Lambda$ having one endpoint in $A_0$ and one endpoint in $B\setminus B_0$, so that $\deg_\Lambda(A_0)=a+b = |B_0|+b$.    Since $\deg_{\Lambda}(v) \le 2$ for each $v\in A_0$ and $\deg_\Lambda(v_0)<2$, $\deg_\Lambda(A_0)<2|A_0|$.  This yields the desired contradiction, since $|N_\Lambda(A_0)| \le \deg_\Lambda(A_0) = |B_0|+b = \deg_\Lambda(A_0)< 2|A_0|$.greater than
\end{proof}

\begin{rem}
  The constant $2$ that appears twice in \cref{hall_vees} can be replaced with any positive integer $c$.
\end{rem}


\section{The Linear Colouring Number of Pseudogrids}


\subsection{Subdivided Grids}

Refer to \cref{grid_figure}.  For positive integers $a$ and $b$, the \emph{$a\times b$ grid} $G_{a\times b}$ is the graph with vertex set $V(G_{a\times b}):=\{1,\ldots,a\}\times\{1,\ldots,b\}$ and that contain an edge with endpoints $(v,w)$ and $(x,y)$ if and only if $|v-x|+|w-y|=1$.  For each $i\in\{1,\ldots,a\}$, the \emph{$i$th column} of $G_{a\times b}$ is the vertex set $\{(i,1),\ldots,(i,b)\}$ and, for each $j\in\{1,\ldots,b\}$, the $j$th row is the vertex set $\{(1,j),\ldots,(a,j)\}$.  We define the \emph{boundary} of $G_{a\times b}$ as $\partial G_{a\times b}:=\{1,a\}\times\{1,\ldots,b\}\cup \{1,\ldots,a\}\times\{1,b\}$.  We say that a path $P:=v_0,v_1,\ldots,v_\ell$ in $G_{a\times b}$ \emph{goes straight} at a vertex $v_i$, $i\in\{1,\ldots,k-1\}$ if $v_{i-1}$, $v_i$, and $v_{i+1}$ are contained in a single row or column.  Otherwise, we say that $P$ \emph{bends} at $v_i$.

\begin{figure}
  \begin{center}
    \begin{tabular}{c@{\hspace{2cm}}c}
      \includegraphics{figs/grid-1} &
      \includegraphics{figs/grid-2}
    \end{tabular}
  \end{center}
  \caption{The grid $G_{10}$ and its $1$-subdivision $G^{\onesub}_{10}$.}\
  \label{grid_figure}
\end{figure}

% An $a'\times b'$ \emph{subgrid} of $G_{a\times b}$ is an induced graph of the form $G_{a\times b}[\{(v,w): v\in\{i,\ldots,i+a'-1\},\, w\in\{j,\ldots,j+b'-1\}$ for some $i\in\{1,\ldots,a-a'+1\}$ and $j\in\{1,\ldots,b-b'+1\}$.   The \emph{vertex boundary} of $G_{a\times b}$ is $\{1,a\}\times\{1,\ldots,b\}\cup \{1,\ldots,a\}\times\{1,b\}$.  The \emph{edge boundary} of $G_{a,b}$ consists of all edges incident to at least one vertex in the vertex boundary of $G_{a,b}$. The \emph{boundary} of $G_{a\times b}$ is the union of the edge and vertex boundaries.

For brevity, we use $G_{k}:=G_{k\times k}$ as a shorthand for the $k\times k$ grid.  We will work extensively with the $1$-subdivision $G^{\onesub}_k$ of $G_k$.  In this case, we define $\partial G^{\onesub}_k:= \partial G_k\cup N_{G^{\onesub}_k}[\partial G_k]$.\todo{Define open and closed neighbourhods $N_G(\cdot)$ and $N_G[\cdot]$.}

\begin{lem}\label{seven_by_seven}
  Let $I,L,Q$ be disjoint subsets of $V(G^{\onesub}_{7})\setminus \partial G^{\onesub}_7$ with $|I\cup L\cup Q|\le 3$ that have the following properties:
  \begin{compactenum}[(i)]\setcounter{enumi}{0}
    \item Each vertex is $I\cup L$ has degree $4$ and each vertex in $Q$ has degree $2$.
    \item No vertex in $V(G^{\onesub}_{10})$ is adjacent to three vertices of $Q$.
    \item No vertex in $I\cup L$ is adjacent to two vertices of $Q$.
  \end{compactenum}
  Then for any pair of boundary vertices $s$ and $t$, $G_{10}$ contains a path $P$ that
  \begin{compactenum}[(a)]
    \item begins at $s$ and ends at $t$;
    \item contains every vertex in $I\cup L\cup Q$;
    \item goes straight through each vertex in $I$; and
    \item bends at each vertex in $L$.
    % \item contains no boundary vertices or edges except $s$ and $t$.
  \end{compactenum}
\end{lem}

\begin{proof}
  I have no idea how to prove this.  Computer search?.
\end{proof}

It will be helpful to take some natural definitions for $G_k$ and extend them to $G^{\onesub}_k$.  To do this, we associate each vertex $v$ of $G^{\onesub}_k$ with a vertex $v^{\leftup}$ of $G_k$. For each vertex $v\in V(G_k)$, we define $v^{\leftup}:=v$. For each subdivision vertex $s$ in $G^{\onesub}_k$ with neighbours $v$ and $w$, we define $s^{\leftup}$ to be the lexicographically smaller of $v$ or $w$. In other words, $s^{\leftup}$ is the vertex of $G_k$ that is directly below or the left of $s$.

For two vertices $v_1,v_2\in V(G^{\onesub}_k)$, with $v^{\leftup}_1:=(x_1,y_1)$ and $v^{\leftup}_2:=(x_2,y_2)$ we define the $L_\infty$-distance between $v_1$ and $v_2$ as $\|v_1v_2\|_\infty = \max\{|x_1-x_2|, |y_1-y_2|\}$. Refer to \cref{ball_figure}.  For each $v\in V(G^{\onesub}_k)$, the \emph{($L_\infty$) ball} of radius $\beta$ centered at $v$ is $B_{\beta}(v):=\{w\in V(G^{\onesub}_k):\|vw\|_\infty \le \beta\}$.  Because it comes up in several places, it is worth noting that $|B_{\beta}(v)|\le 4(\beta+1)^2$, and this bound is tight for each $v\in\{\beta+1,\ldots,\beta-r-1\}\times\{\beta+1,\ldots,b-\beta-1\}$.

\begin{figure}
  \begin{center}
    \includegraphics{figs/grid-3}
  \end{center}
  \caption{The ball $B_2(v)=B_2(s)$ in $G^{\onesub}_{10}$.}
  \label{ball_figure}
\end{figure}


\begin{lem}\label{four_points}
  Let $P$ be a set of at most $3$ vertices of $G^{\onesub}_k$.  Then there exists a set $S$ of pairwise-disjoint balls, each of radius $3$, such that each vertex in $P$ is contained in the interior of some ball in $S$.\todo{Define the interior of a ball.}
\end{lem}

\begin{proof}
  Let $P:=\{v_1,\ldots,v_t\}$ where $p_i^{\leftup}:=(x_i,y_i)$.  Let $X:=\bigcup_{i=1}^t \{x_i\bmod 7,(x_{i+1})\bmod 7\}$ and let $Y:=\bigcup_{i=1}^t\{y_i\bmod 7,(y_i+1)\bmod 7\}$.  Then $|X|,|Y|\le 2t\le 6$.  In particular, there exists some $x\in\{0,\ldots,6\}\setminus X$ and some $y\in\{0,\ldots,6\}\setminus Y$. For each $i,j\in\Z$, let $c_{i,j}:=(7i+x+3,7j+y+3)$ and observe that the ball $B_{3}(c_{i,j})$ does not have any point of $P$ on its boundary.  Furthermore, the set of balls $S:=\{B_{3}(c_{i,j}):i,j\in\Z\}$ is a partition of $V(G^{\onesub}_{k})$.  Therefore $S$ is a set of pairwise-disjoint balls, each of radius $3$, such that each vertex in $P$ is contained in the interior of some ball in $S$.
\end{proof}

\begin{lem}
  Let $P$ be a subset of $V(G^{\onesub}_k)$ such that no ball of radius $18$ contains more than $3$ vertices in $P$.  Then there exists a set $S$ of disjoint balls, each of radius $3$ and such that each point of $P$ is in the interior of some ball in $S$.
\end{lem}

\begin{proof}
  Define a graph $H$ with vertex set $V(H):=P$ and that contains an edge $pq$ if and only if $\|pq\|_\infty \le 12$. Observe that, if some connected subgraph of $H$ contains $4$ vertices, then these $4$ vertices are contained in a ball of radius $18$.  Therefore each component of $H$ contains at most $3$ vertices.

  For each component $C$ of $H$, apply \cref{four_points} to $V(C)$ and keep only those balls in $S$ that contain at least one vertex of $V(C)$.  This produces a set $S_C$ of disjoint balls, each of radius $3$, that each contain some vertex in $V(C)$ in their interior.  Any two balls $B_{C_1}$ and $B_{C_2}$ obtained from different components of $H$ are disjoint because the first ball contains a vertex $p_1\in P$ and the second ball contains a vertex $p_2\in P$ such that $\|p_1p_2\|\infty > 12$, but the balls $s_1$ and $s_2$ each have radius $3$.
\end{proof}


\begin{lem}\label{one_path_enchilada}
  Let $k$ be a positive integer, let $I,L,Q$ be disjoint subsets of $V(G^{\onesub}_{k})\setminus\partial G^{\onesub}_k$ that have the following properties:
  \begin{compactenum}[(1)]\setcounter{enumi}{-1}
    \item Any ball of radius 18 contains at most $3$ elements of $I\cup L\cup Q$.
    \item Each vertex in $I\cup L$ has degree $4$ and each vertex in $Q$ has degree $2$.
    \item No vertex in $V(G^{\onesub}_{10})$ is adjacent to three vertices of $Q$.
    \item No vertex in $I\cup L$ is adjacent to two vertices of $Q$.
  \end{compactenum}
  Then $G^{\onesub}_k$ contains a path $P$ that
  \begin{compactenum}[(a)]
    \item contains all the vertices in $I\cup L\cup Q$;
    \item goes straight through each vertex in $I$; and
    \item turns at each vertex in $L$.
  \end{compactenum}
\end{lem}

\begin{proof}
  Start with the snakelike path $P_0$ that visits every $7$th row of $G_k$. Cover the elements of $I\cup L\cup Q$ with disjoint balls of side length $3$.  Each such ball is intersected by exactly one row of the snakelike path.  Use \cref{seven_by_seven} to locally deform $P_0$ to pick up the objects in each ball that it intersects.
\end{proof}


\subsection{Approximate Voronoi Diagrams in $G^{\onesub}_k$}


% \todo[inline]{This lemma is not true. Pick a center $x$ and place a point in each of the quadrants around $x$ where the $i$th point is at distance $10\beta - (ir)$ from $x$.  Then $x$ is incident on all four Voronoi regions.  We really have to use the fact that $\beta$ (or rather, $\beta+ar$) is also an upper bound on the distance we actually care about.}
\begin{lem}\label{voronoi_mapping}
  Let $\beta, k\in\N$, let $P\subseteq V(G^{\onesub}_k)$ be such that $\min\{\|pq\|_\infty:p,q\in\binom{P}{2}\} \ge \beta$, and define  $Q:=\bigcup_{p\in P} B_\beta(p)$. Then there exists a function $\vor:Q\to P$ such that
  \begin{compactenum}[(i)]
    \item \label{real_close} For each $p\in P$ and each $v\in B_{\beta/8-1}(p)$, $\vor(v)=p$.
    \item \label{never_far} For each $v\in Q$, $\|v\vor(p)\|_\infty \le 7\beta/4$.
    \item \label{three_in_ball} For any ball $B$ of radius $\beta/12$, $|\{\vor(v):v\in B\cap Q\}|\le 3$.
    \item \label{twocolour_neighbourhood} For any $v\in Q$, $|\{\vor(v)\}\cup\{\vor(w):w\in Q\cap N_{G^{\onesub}_k}(v)\}|\le 2$.
  \end{compactenum}
\end{lem}

\begin{proof}
  We first define a function $\vor':Q\to P$ that satisfies (\ref{real_close})--(\ref{three_in_ball}) and then show how local modifications can be used to obtain a function $\vor$ that also satisfies (\ref{twocolour_neighbourhood}).

  Arbitrarily order the vertices of $P$ as $p_1,\ldots,p_{|P|}$.  For each $v\in Q$, let $d(v):=\min\{\|vp\|_\infty:p\in P\}$, let $S(v):=\{p\in P:\|vp\|_\infty \le d(v)+3\beta/4\}$, let $i(v):=\min\{i:p_i\in S(v)\}$, and let $\vor'(v)=p_{i(v)}$.  In words, $d(v)$ represents the distance from $v$ to its nearest neighbour in $P$, $S(v)$ represents the set of approximate nearest neighbours of $v$ in $P$ (up to an additive error of $\beta/12$), and $\vor'(v)$ is the approximate nearest neighbour of $v$ of smallest index.

  To see that $\vor'$ satisfies (\ref{real_close}), observe that, for each distinct $p,q\in P$ and $v\in B_{\beta/8-1}(p)$, the triangle inequality implies that $\|vq\|_\infty \ge \|pq\|_\infty-\|vp\|_\infty \ge \beta-\beta/8+1 > 7\beta/8$.  On the other hand $d(v)\le \beta/8$, so $d(v)+3\beta/4\le 7\beta/8$.  Therefore, $q\not\in S(v)$ for any $q\in P\setminus\{p\}$, so $\vor'(v)=p$.

  That $\vor'$ satsifies (\ref{never_far}) follows immediately from the fact that each $v\in Q$, so $d(v)\le \beta$, and therefore $\|pv\|_\infty\le 7\beta/4$ for each $p\in S(v)$.

  We now argue that $\vor'$ satisfies (\ref{three_in_ball}).  Let $v_1,\ldots,v_4\in Q$ and $1\le i_1<\cdots<i_4\le|P|$ be such that $\vor'(v_j)=p_{i_j}$ for each $j\in\{1,\ldots,4\}$.  Suppose, for the sake of contradiction, that $i_1,\ldots,i_4$ are all distinct and that $v_1,\ldots,v_4$ are contained in some ball $B_{\beta/12}(c)$.  By the triangle inequality $\|v_iv_j\|_\infty \le \|v_ic\|_\infty+\|cv_j\|_\infty \le \beta/6$ for each  $i,j\in\{1,\ldots,4\}$.
  Since $v_1\in Q$, $d(v)\le\beta$ so
  \[
    \|p_{i_1}v_1\|_\infty \le d(v_1) + 3\beta/4 \le \beta + 3\beta/4 \enspace .
  \]
  Since $\vor'(v_2)=p_{i_2}\neq p_{i_1}$,
  \[
    \|p_{i_2}v_2\|_\infty
     < \|p_{i_1}v_2\|_\infty - 3\beta/4
     \le \|p_{i_1}v_1\|_\infty + \beta/6 - 3\beta/4
     \le \beta + \beta/5 \enspace .
  \]
  Since $\vor'(v_3)=p_{i_3}\neq p_{i_2}$,
  \[
     \|p_{i_2}v_3\|_\infty
     < \|p_{i_2}v_3\|_\infty - 3\beta/4
     \le \|p_{i_2}v_2\|_\infty + \beta/6 - 3\beta/4
     < \beta + 2\beta/6 - 3\beta/4 \enspace .
  \]
  Finally, since $\vor'(v_4)=p_{i_4}\neq p_{i_3}$,
  \[
    \|p_{i_4}v_4\|_\infty
    < \|p_{i_3}v_4\|_\infty - 3\beta/4
    \le \|p_{i_3}v_3\|_\infty + \beta/5 - 3\beta/4
    < \beta + 3\beta/6 - 3\beta/2 = 0 \enspace ,
  \]
  which is clearly a contradiction.  This establishes that $\vor'$ satisfies Condition~(\ref{three_in_ball}).\todo{TODO: Optimize and cleanup. Finish proving (iii)}
\end{proof}

\subsection{Pseudogrids}

A graph $H$ is an $a\times b$ \emph{pseudogrid} if there exists a set $\mathcal{P}$ of vertex-disjoint paths in $H$ such that the quotient graph $H':=H/\mathcal{P}$ is isomorphic to a subdivision of $G_{a\times b}$.  The following lemma connecting grid minors and pseudogrids is stated implicitly by \citet{kun.obrien.ea:polynomial}.  We give a proof here because some of the details of the connection are required in order to prove our results.


\begin{lem}\label{pseudogrid_minor}
  If a graph $G$ contains a $G_{a\times b}$ grid minor then $G$ contain a subgraph $H$ that is an $a\times b$ pseudogrid.
\end{lem}

\begin{proof}
  (This proof only uses the fact that $G_{a\times b}$ has maximum degree  $4$.)  Since $G$ contains a $G_{a\times b}$ minor then, by definition, $G$ contains a subgraph $H$ and a partition $\mathcal{B}:=(B_x:x\in V(G_{a\times b}))\}$ of $V(H)$ such that
  \begin{inparaenum}[(i)]
    \item $B_x$ is non-empty and $H[B_x]$ is connected for each $x\in V(G_{a\times b})$;
    \item $H/\mathcal{B}$ is isomorphic to $G_{a\times b}$ using the isomorphism given by $B_x\mapsto x$.
  \end{inparaenum}
  Let $(H,\mathcal{B})$ be chosen so that $|E(H)|$ is minimum, over all choices of $(H,\mathcal{B})$ that satisfy (i) and (ii).

  An edge of $vw$ of $H$ is a \emph{border edge} if $v\in B_x$ and $w\in B_y$ for distinct $x,y\in V(G_{a\times b})$.  For each $x\in V(G_{a\times b})$, let $\partial B_x$ denote the set of border edges incident on one vertex of $B_x$.  For each $x\in V(G_{a\times b})$, let $T_x:= H[B_x]\cup \partial B_x$.   Since $H$ is edge-minimal, $|\partial B_x|=\deg_{G_{a\times b}}(x)\le 4$, for each $x\in V(G_{a\times b})$ and $T_x$ is a tree with exactly $\deg_{G_{a\times b}}(x)$ leaves.\todo{Check the preceding claim.}  We define $\mathcal{P}:=(P_x:x\in V(G_{a\times b}))$ as follows:
\begin{compactenum}
  \item \label{one_special} If $T_x$ has exactly one node $v$ of degree at least three, then $P_x:=\{v\}$.
  \item \label{two_degree_three} If $T_x$ has two nodes $v$ and $w$ each of degree $3$ then $P_x$ contains all vertices on the path in $T_x$ from $v$ to $w$.
  \item \label{just_a_path} Otherwise $T_x$ is a path $v_0,\ldots,v_\ell$ for some $\ell\ge 2$ and $P_x:=v_1$.
\end{compactenum}
It is straightforward to verify that, with this assignment $\mathcal{P}:=(P_x:x\in V(G_{a\times b}))$, the quotient graph $H/\mathcal{P}$ is a subdivision of $G_{a\times b}$, so that $H$ is indeed an $a\times b$ pseudogrid.
\end{proof}

The second case (\ref{two_degree_three}) in the proof of \cref{pseudogrid_minor} occurs when $\deg_{G_{a\times b}}(x)=4$, so $x=(i,j)$ for some $i\in\{2,\ldots,a-1\}$ and $j\in\{2,\ldots,b-1\}$.  This case is especially important in our proof because it corresponds to a case in which some of the vertices of $P_x$ are shared by the $i$th column and the $j$th row of the pseudogrid. However, a path in the pseudogrid that uses vertices of $B_x$ may or may not contain all the vertices of $P_x$ depending on whether or not the corresponding path in $G_{a\times b}$ bends at $x$.  This motivates the following definitions (see \cref{straight_or_bent}):
\begin{compactenum}
  \item We say that the $x$ is \emph{straight} if the path from $B_{(i-1,j)}$ to $B_{i+1,j}$ in $T_x$ contains all the vertices of $P_x$.
  \item Otherwise, we say that $x$ is \emph{bent}.
\end{compactenum}
\begin{figure}
  \begin{center}
    \begin{tabular}{c@{\hspace{1cm}}c}
      \includegraphics{figs/bx-paths-1.pdf} & \includegraphics{figs/bx-paths-3} \\
      (a) & (b)
    \end{tabular}
  \end{center}
  \caption{The vertex $x:=(i,j)$ is (a)~straight if the path from $B_{(i,j-1)}$ to $B_{(i+1,j)}$ uses all vertices of $P_x$ or (b)~bent otherwise.}
  \label{straight_or_bent}
\end{figure}
Note that, if $x$ is straight, then the path from $B_{i,j-1}$ to $B_{i,j+1}$ in $T_x$ also contains all the vertices of $P_x$.  Furthermore, if $x$ is bent, then, for each $\overline\i,\overline\j\in\{-1,1\}$, the path from $B_{i,j+\overline\j}$ to $B_{i+\overline\i, j}$ in $T_x$ contains all the vertices of $P_x$.


\todo[inline]{I can't yet prove the full result, which is why the next lemma has some weird conditions.}

\begin{lem}
  %For any $k\times k$ pseudogrid $H$
  Let $H$ be a $k\times k$ pseudogrid and $\mathcal{P}:=(P_x:x\in V(G_{k}))$ be such that
  \begin{inparaenum}[(i)]
      \item for each $x\in V(G_k)$, $H[P_x]$ is a path of length at most $r$ and
      \item $H/\mathcal{P}$ is a $\le\!\! r$-subdivision of $G_k$.
  \end{inparaenum}
  Then $\chilin(G)\in\Omega(k/r)$.
\end{lem}

\begin{proof}
  This proof uses two large integer constants $\alpha$ and $\beta$ and two small positive real constants $\epsilon$ and $\delta$.  The relationships between these constants are that $\beta = \Theta(\alpha^{1/4})$ and $\epsilon=\Theta(1/\alpha)$.  Let $\varphi:V(H)\to\{1,\ldots,C\}$ be an arbitrarily $C$-colouring of some $k\times k$ pseudogrid $H$ for some $C\le \epsilon k$.  We will show that $H$ contains a path $P$ which has no centre with respect to $\varphi$.

  First we introduce some notations for vertices of the subdivided grid $G^{\onesub}_k$ to their corresponding paths in the pseudogrid $H$.  For each vertex $v$ of $G_k$ there is a path $P_v$ in $\mathcal{P}$ (possibly consisting of one vertex) such that the vertex $v$ appears in $H'$ as a result of contracting $P_v$ in $H$.  In this way, each vertex $v$ of $G_k$ is associated with a non-empty \emph{colour set} $\Phi(v):=\{\varphi(w):w\in V(P_v)\}$.  Similarly, for each subdivision vertex $s$ of $G^{\onesub}_k$ that comes from subdividing an edge $e$ of $G_k$, there is a path $P_s$ in $H'$ that comes from subdividing $e$ (possibly zero times).  Thus, each subdivision vertex $s$ of $G^{\onesub}_k$ is associated with a (possibly empty) \emph{colour set} $\Phi(s):=\{\varphi(w):w\in P^-_e\}$, where $P^-_e$ denotes the (possibly empty) set of internal vertices on the path $P_e$.

  For each colour $c\in\{1,\ldots,C\}$, define the \emph{multiplicity} of $c$ as the number of vertices $G^{\onesub}_k$ that include $c$ in their colour set.  Fix some large integer constant $\alpha$ and say that $\alpha$ is \emph{infrequent} if it has multiplicity less than $\alpha$ and \emph{frequent} otherwise.

  \begin{clm}\label{all_frequent}
    There exists an integer $k'\ge (1-\epsilon)k$ such that
    $H$ contains a $k'\times k'$ pseudogrid $H'$ in which every colour in $\{\varphi(v):v\in V(H')\}$ is frequent.
  \end{clm}

  \begin{proof}
    This proof uses the notion of \emph{removing} a row or column of a pseudogrid.  Removing a row is defined as follows (removing a column is similar): For each vertex $v$ of $G^{\onesub}_k$ in the relevant row, remove all edges of $P_v$ from $H$. Next, remove any isolated vertices from $H$ and finally, repeatedly remove degree-$1$ vertices from $H$ until none remain.  The result is pseudogrid with one less row.

    While there exists some infrequent colour $c\in\{1,\ldots,C\}$, find a row or column of $G^{\onesub}_k$ that contains a vertex with $c$ in its colour set and remove that row or column.  Doing this exhaustively results in the removal of at most $(\alpha-1) C \le \epsilon\alpha k$ rows and at most $\epsilon\alpha k$ columns of $H$.  This leaves a pseudogrid $H'$ that contains a $k'\times k'$ pseudogrid for some $k'\ge (1-\epsilon \alpha)k$ for which every colour that appears in at least one vertex of $H'$ is frequent.
  \end{proof}

  Because of \cref{all_frequent}, we now assume that there is no infrequent colour, i.e., each colour $c\in\{1,\ldots,C\}$ has multiplicity at least $\alpha$.

  % Colour sets turn out to be problematic, so we would like to assign a single colour $c\in\Phi(v)$ to each $v\in G^{\onesub}_k$.  To do this, consider the bipartite graph $X$ with parts $A:=\{1,\ldots,C\}$ and $B:=V(G^{\onesub}_k)$ and which contains the edge $cv$ if and only if $c\in\Phi(v)$.  Since each colour in $A$ is frequent, $\deg_x(c)\ge a$ for each $c\in C$.  S

  Next we run a two stage process that, for each $c\in\{1,\ldots,C\}$, will identify a pair $(x_c,y_c)$ where each of $x_c$ and $y_c$ is vertex of $G^{\onesub}_k$ that contains $c$ in its colour set.  These pairs are carefully chosen so that we will be able to construct a simple path $P_k$ in $G^{\onesub}_k$ that contains every vertex in $X:=\bigcup_{c\in\{1,\ldots,C\}} \{x_c,y_c\}$.  Furthermore, this path will bend at every bent vertex in $X$ and go straight at every straight vertex in $X$.  This will ensure that the corresponding path in $H$ contains at least two vertices of each colour.

  \begin{enumerate}[{Stage} 1:]
    \item We begin this stage by defining each vertex $G^{\onesub}_k$ as being \emph{unmarked}.  Next we consider each $c\in\{1,\ldots,C\}$ in turn.  If $G^{\onesub}_k$ contains a pair $(x_c,y_c)$ where
    \begin{inparaenum}[(i)]
      \item each of $x_c$ and $y_c$ is unmarked;
      \item $\|x_cy_c\|_\infty \ge \beta$; and
      \item $c\in \phi(x_c)$ and $c\in \phi(y_c)$;
    \end{inparaenum}
    then we say that Stage~1 \emph{succeeeds} for colour $c$ and we mark each vertex in $B_{\beta}(x_c)$ and $B_\beta(y_c)$.

    If $G_k$ contains no such pair $(x_c,y_c)$, then we leave $x_c$ and $y_c$ undefined for now and say that Stage~1 \emph{fails} for colour $c$.  Note that Stage~1 can only fail for $c$ if $G^{\onesub}_k$ contains fewer than $(2\beta+1)^2$ unmarked vertices with $c$ in their colour sets.  In particular, for $\alpha > (2\beta+1)^2$, Stage~1 succeeds for $c=1$.

    \item Let $S\subset\{1,\ldots,C\}$ be the set of colours for which Stage~1 succeeds and let $\overline{S}:=\{1,\ldots,C\}\setminus S$ be the set of colours for which Stage~1 fails.  Let $R_1:=\bigcup_{c\in S}\{x_c,y_c\}$ be the set of vertices and edges chosen in $R_1$.  By construction, $\min\{\|pq\|_\infty: \{p,q\}\in \binom{R_1}{2}\}\ge \beta$.

    As mentioned above, for sufficiently large $\alpha$ (relative to $\beta$), the set $S$ is non-empty and therefore $R_1$ is non-empty.  Let $P:=R_1$ and let $Q:=\bigcup_{p\in P} B_\beta(p)$ and apply \cref{voronoi_mapping} to obtain the mapping $\vor:Q\to P$.  We use to this to construct a bipartite graph $I$ with vertex parts $\overline{S}$ and $R_1$. The edge set of $I$ contains an edge $cx$ precisely when there is a marked vertex $z$ of $G^{\onesub}_k$ such that $c\in\phi(z)$ and $\vor(z)=x$.

    First observe that each vertex $x$ of $G^{\onesub}_k$ corresponds to a path $P_x$ in $H$ that contains at most $r$ vertices.\footnote{For some subdivision vertices of $G^{\onesub}_k$, $P_x$ may be empty.}  This implies that $|\Phi(x)|\le r$ for each $x\in V(G^{\onesub}_k)$.  Next observe that, for each $p\in R_1$, \cref{voronoi_mapping}(\ref{never_far}) implies that $|\vor^{-1}(p)|\le|B_{7\beta/4}(p)|\le (7\beta/2+1)^2$.  This implies that $\deg_I(p) \le r(7\beta/2+1)^2$ for each $p\in R_1$.

    This implies that, for any $A_0\subseteq\overline{S}$,
    \[
        |N_I(A_0)| \ge \frac{a|A_0|}{r(7\beta/2+1)^2} \ge 2|N_I(A_0)| \enspace .
    \]
    for appropriately chosen $\alpha$ and $\beta$ (with $\beta\in O(\sqrt{\alpha}))$.\todo{Specify precise condition.}  Therefore, by \cref{hall_vees}, there exists a subgraph $\Lambda$ of $I$ in which each vertex of $A$ has degree $2$ and each vertex of $B$ has degree at most one.  Therefore, each $c\in A=\overline{S}$ has two $\Lambda$-neighbours $p$ and $q$ in $B$.  The edge $cp$ is present in $I$ because $G_k$ contains a vertex $x_c$ with $s(x_c)=p$ and $c\in \Phi(x_c)$.  Similarly, the edge $cq$ is present in $I$ because $G_k$ contains a vertex $y_c$ with $s(y_c)=q$ and $c\in\Phi(y_c)$.  In this way, the graph $\Lambda$ defines the pairs $\{x_c,y_c\}$ for each $c\in\overline{S}$.
  \end{enumerate}

  Summarizing our work thus far, we have identified a subset  $R:=\bigcup_{c=1}^C\{x_c,y_c\}$ of at most $2c$ vertices and edges of $G_k$ where $c\in \Phi(x_c)\cap \Phi(y_c)$ for each $c\in\{1,\ldots,C\}$.
  Let $R_1:=z\in\bigcup_{c\in S}\{x_c,y_c\}$ and $R_2:=z\in\bigcup_{c\in \overline{S}}\{x_c,y_c\}$ be the subsets of $R$ obtained during Stages~1 and 2, respectively.

  \todo[inline]{There is a technical condition missing here.  To apply \cref{one_path_enchilada} we need the elements of $R$ to be distance at least $3$ from the boundary of $G^{\onesub}_k$.  When counting multiplicity, we shouldn't count occurrences of a colour that are close to the boundary.}

  Conditions~(\ref{never_far}) and (\ref{three_in_ball}) of \cref{voronoi_mapping} imply that each any ball of radius $\min\{\beta/8-1, \beta/12\}/2$ contains at most three elements of $R$.  Therefore, (for sufficiently large $\beta$) \cref{one_path_enchilada}, there exists a path $P$ in $G^{\onesub}_k$ that contains every vertex of $R$, that goes straight at every straight vertex of $R$ and that bends at every bent vertex of $R$. These latter two conditions guarantee that $V(P)$ contains $P_x$ for each $x\in R$. For each colour $c\in\{1,\ldots,C\}$, there are at least two vertices $x_c,y_c\in R$ such that $P_{x_c}$ and $P_{y_c}$ each contain some vertex of colour $c$.  Therefore the path $P$ contains two vertices of each colour.  Therefore $P$ has no center. Therefore $\varphi$ is not a linear centered colouring of $H$.  Since $\varphi$ is an arbitrary $\epsilon k$-colouring of $H$, this implies that $\chilin(H)\ge \epsilon k$.
\end{proof}







\bibliographystyle{plainurlnat}
\bibliography{lin-vs-cen}




\end{document}
