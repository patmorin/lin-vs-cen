\documentclass{patmorin}
\listfiles
\usepackage{pat}
\usepackage{paralist}
\usepackage{dsfont}  % for \mathds{A}
\usepackage[utf8x]{inputenc}
\usepackage{skull}
\usepackage{paralist}
\usepackage{graphicx}
\usepackage[noend]{algorithmic}

\usepackage[normalem]{ulem}
\usepackage{cancel}
\usepackage{enumitem}

\usepackage{todonotes}

\usepackage[longnamesfirst,numbers,sort&compress]{natbib}

\usepackage[mathlines]{lineno}
\setlength{\linenumbersep}{2em}
% \linenumbers
% \rightlinenumbers
% \linenumbers
\newcommand*\patchAmsMathEnvironmentForLineno[1]{%
 \expandafter\let\csname old#1\expandafter\endcsname\csname #1\endcsname
 \expandafter\let\csname oldend#1\expandafter\endcsname\csname end#1\endcsname
 \renewenvironment{#1}%
    {\linenomath\csname old#1\endcsname}%
    {\csname oldend#1\endcsname\endlinenomath}}%
\newcommand*\patchBothAmsMathEnvironmentsForLineno[1]{%
 \patchAmsMathEnvironmentForLineno{#1}%
 \patchAmsMathEnvironmentForLineno{#1*}}%
\AtBeginDocument{%
\patchBothAmsMathEnvironmentsForLineno{equation}%
\patchBothAmsMathEnvironmentsForLineno{align}%
\patchBothAmsMathEnvironmentsForLineno{flalign}%
\patchBothAmsMathEnvironmentsForLineno{alignat}%
\patchBothAmsMathEnvironmentsForLineno{gather}%
\patchBothAmsMathEnvironmentsForLineno{multline}%
}


\newcommand{\coloured}[2]{{\color{#1}{#2}}}
\newenvironment{colourblock}[1]{\color{#1}}{}

\newcommand{\condref}[1]{(C\ref{#1})}

% Taken from
% https://tex.stackexchange.com/questions/42726/align-but-show-one-equation-number-at-the-end
\newcommand\numberthis{\addtocounter{equation}{1}\tag{\theequation}}


\setlength{\parskip}{1ex}


\DeclareMathOperator{\diam}{diam}
\DeclareMathOperator{\tw}{tw}
\DeclareMathOperator{\stw}{stw}
\DeclareMathOperator{\ltw}{ltw}
\DeclareMathOperator{\pw}{pw}
\DeclareMathOperator{\lpw}{lpw}
\DeclareMathOperator{\lhptw}{lhp-tw}
\DeclareMathOperator{\lhppw}{lhp-pw}

\DeclareMathOperator{\x}{x}
\DeclareMathOperator{\depth}{d}
\DeclareMathOperator{\sh}{cbt}
\DeclareMathOperator{\cbt}{cbt}
\DeclareMathOperator{\sgn}{sgn}
\DeclareMathOperator{\dc}{dc}

\DeclareMathOperator{\afci}{\overline{\chi}_\pi}
\DeclareMathOperator{\afcn}{\dot{\chi}_\pi}

\newcommand{\ellt}{{\lfloor\ell/2\rfloor}}

\title{\MakeUppercase{Linear versus centred Colouring Numbers}\thanks{This research was partly funded by NSERC.}}
\author{Prosenjit Bose, Vida Dujmović, Mehrnoosh Javarsineh, and Pat Morin}

\date{}

% \DeclareMathOperator{\ddiv}{div}
% \DeclareMathOperator{\hist}{p}
\DeclareMathOperator{\dist}{dist}

\newcommand{\colored}[2]{{\color{#1}#2}}

\usepackage{tabularx}

\DeclareMathOperator{\ci}{\overline{\pi}}

\DeclareMathOperator{\chicen}{\chi_{\mathrm{cen}}}
\DeclareMathOperator{\chilin}{\chi_{\mathrm{lin}}}

\begin{document}

% \begin{titlepage}
\maketitle

\begin{abstract}
    These are some notes on the relationships betwee linear colouring numbers and centred colouring numbers.
\end{abstract}
% \end{titlepage}

% \pagenumbering{roman}
% \tableofcontents
%
% \newpage
% \pagenumbering{arabic}



\section{Introduction}

Let $G$ be a simple undirected graph.  A \emph{$k$-colouring} of $G$ is any function $\varphi:V(G)\to S$ where $S$ is a set of size $k$.  A vertex $v$ of $G$ is a \emph{centre} of $G$ with respect to $\varphi$ if $\varphi(v)\not\in\{\varphi(w):w\in V(G)\setminus\{v\}\}$, i.e., $v$ is the unique vertex of $G$ having colour $\varphi(v)$.  A colouring $\varphi$ of $G$ is \emph{centred} if every connected subgraph of $G$ has a centre with respect to $\varphi$. A colouring $\varphi$ of $G$ is \emph{linear} if every simple path in $G$ has a centre with respect to $\varphi$.

The \emph{centred chromatic number} $\chicen(G)$ of $G$ is the minimum integer $c$ such that $G$ has a centred $c$-colouring.  The \emph{linear chromatic number} of $G$ is the minimum integer $\ell$ such that $G$ has a linear $\ell$-colouring.  Since every path in $G$ is a connected subgraph of $G$, every centred colouring of $G$ is also a linear colouring of $G$, so $\chilin(G)\le\chicen(G)$.

The question of upper bounding $\chicen(G)$ by some function of $\chilin(G)$ was considered by \citet[Theorem~1]{kun.obrien.ea:polynomial} who showed the following result:

\begin{thm}[\citet{kun.obrien.ea:polynomial}]\label{kun-obrien-general}
  For any graph $G$, $\chicen(G)\le \chilin^{190}(G)\cdot\log^{O(1)}(\chilin(G))$.
\end{thm}


Their proof of \cref{kun-obrien-general} has three steps:
\begin{enumerate}
  \item A theorem of \citet{kawarabayashi.rossman:polynomial} shows that, if $\chicen(G)\ge k^{190}\log^{O(1)} k$ then $\chilin(G)\ge k^{38}$ or $G$ has treewidth $\tw(G)\ge k^{38}$.  In the former case there is nothing left to prove.
  \item The current-best version of the Excluded Grid Theorem due to \citet{chuzhoy:improved} shows that, if $\tw(G)\ge k^{38}$, then $G$ contains an $\Omega(k^2\times k^2)$ grid minor.
  \item A technical lemma \cite[Lemma~5]{kun.obrien.ea:polynomial} shows that, if $G$ contains a $k^2\times k^2$ grid minor, then $\chilin(G)\in\Omega(k)$.
\end{enumerate}

These notes are an attempt to improve the bound in \cref{kun-obrien-general} in the general case as well as for some special classes of graphs.

As a first step, we attempt to improve the third part of the argument to show that, if $G$ contains a $k\times k$ grid minor, then $\chilin(G)\in \Omega(k)$.  This, by itself, reduces the exponent in \cref{kun.obrien.general} from $190$ to $95$.  Next, we observe that the first two parts of the argument are both ``heavy hammers'' and that, for some special graph classes, much lighter hammers suffice.  For example, if we consider only planar graphs, then it is well known that any planar graph $G$ of treewidth $k$ contains an $\Omega(k\times k)$ grid minor.  This already reduces the exponent further (for planar graphs) to $5$.  If one could prove a similar result for the first ``heavy hammer'' then this would show that, for any planar graph $G$, $\chicen(G)\in O(\chilin(G))$.


\section{Preliminaries}

For a graph $G$ and a vertex $v\in V(G)$, $\deg_G(v):=|\{vw\in E(G)\}|$ denotes the degree of $v$ in $G$.  For a any $S\subseteq V(G)$, $\deg_G(S):=\sum_{v\in S}\deg_G(v)$ is the total degree of $S$.\todo{Surely this next lemma is well-known?}

\begin{lem}\label{hall_vees}
  Let $G$ be a bipartite graph with vertex parts $A$ and $B$ and having the property that, for any $A_0\subseteq A$, $|N_G(W)|\ge 2|A_0|$.  Then $G$ contains a subgraph $\Lambda$ in which each vertex of $A$ has degree exactly two and each vertex of $B$ has degree at most one.
\end{lem}

\begin{proof}
  Let $\Lambda$ be a subgraph of $G$ in which each vertex of $A$ has degree at most two, each vertex of $B$ has degree at most one, and that maximizes $\deg_\Lambda(A)$.  If $\deg_{\Lambda}(A) = 2|\Lambda|$ then there is nothing to prove. Assume therefore, for the sake of contradiction,  that there exists $v_0\in A$ with $\deg_\Lambda(v_0) < 2$.

  We say that a path $v_0,\ldots,v_m$ in $G$ is \emph{$\Lambda$-aternating} if $v_iv_{i+1}\in E(\Lambda)$ for all odd $i\in\{0,\ldots,m-1\}$ and $v_iv_{i+1}\not\in E(\Lambda)$ for all even $i\in\{0,\ldots,m\}$.  Observe that any such path has even length so that it ends at a vertex $v_m\in A$.  Otherwise, replacing the edges $\lfloor m/2\rfloor$ edges $\{v_1v_2,v_3v_4,\ldots,v_{m-2}v_{m-1}\}\in E(\Lambda)$ with the $\lceil m/2\rceil$ edges $\{v_0v_1,v_2v_3,\ldots,v_{m-1}v_m\}$ does not change $\deg_\Lambda(v)$ for any $v\in A_0\setminus\{v_0\}$ but does increase  $\deg_\Lambda(v_0)$, contradicting the assumption that $\Lambda$ maximizes $\deg_\Lambda(A)$.

  Let $A_0\subseteq A$ and $B_0\subseteq B$ be the subsets of $A$ and $B$, respectively, that can be reached from $v_0$ by $\Lambda$-alternating paths.
  Observe that, for any $x\in B_0$,  there is exactly one edge $vx$ of $\Lambda$ incident on $x$ and the other endpoint $v$ of this edge is in $A_0$.  Let $a$ be the number of edges of $\Lambda$ having one endpoint $A_0$ and one endpoint in $B_0$, so that $a=\deg_\Lambda(B_0)=|B_0|$.  Let $b$ be the number of edges of $\Lambda$ having one endpoint in $A_0$ and one endpoint in $B\setminus B_0$, so that $\deg_\Lambda(A_0)=a+b = |B_0|+b$.    Since $\deg_{\Lambda}(v) \le 2$ for each $v\in A_0$ and $\deg_\Lambda(v_0)<2$, $\deg_\Lambda(A_0)<2|A_0|$.  This yields the desired contradiction, since $|N_\Lambda(A_0)| \le \deg_\Lambda(A_0) = |B_0|+b = \deg_\Lambda(A_0)< 2|A_0|$.
\end{proof}


\section{The Linear Colouring Number of Pseudogrids}

A \emph{subdivision} $G'$ of a graph $G$ is any graph that can be obtained from $G$ by repeatedly replacing some edge edge $uw$ of $G$ with a path $uvw$, where $v\not\in V(G)$ is a newly introduced degree-$2$ vertex.  Alternatively, $G'$ is obtained from $G$ by replacing some of the edges of $G$ with paths whose internal vertices have degree $2$.

For positive integers $a$ and $b$, the $a\times b$ \emph{grid} $G_{a\times b}$ is the graph with vertex set $V(G_{a\times b}):=\{1,\ldots,a\}\times\{1,\ldots,b\}$ and that contain an edge with endpoints $(v,w)$ and $(x,y)$ if and only if $|v-x|+|w-y|=1$. For brevity, we use $G_{k}:=G_{k\times k}$ as a shorthand for the $k\times k$ grid.  An $a'\times b'$ \emph{subgrid} of $G_{a\times b}$ is an induced graph of the form $G_{a\times b}[\{(v,w): v\in\{i,\ldots,i+a'-1\},\, w\in\{j,\ldots,j+b'-1\}$ for some $i\in\{1,\ldots,a-a'+1\}$ and $j\in\{1,\ldots,b-b'+1\}$.   The \emph{vertex boundary} of $G_{a\times b}$ is $\{1,a\}\times\{1,\ldots,b\}\cup \{1,\ldots,a\}\times\{1,b\}$.  The \emph{edge boundary} of $G_{a,b}$ consists of all edges incident to at least one vertex in the vertex boundary of $G_{a,b}$. The \emph{boundary} of $G_{a\times b}$ is the union of the edge and vertex boundaries.


\begin{lem}
  Let $I,L$ be disjoint subsets of $V(G_{10})$, let $Q$ be a subset of $E(G_{10})$ and let $X:=I\cup L\cup Q$ have the following properties:
  \begin{compactenum}[(Pr1)]\setcounter{enumi}{0}
    \item $|X|\le 4$.
    \item $Q$ does not contain the edges of a $4$-cycle.
    \item No vertex of $G_k$ is incident to three or more edges in $Q$.
    \item No vertex in $I\cup L$ is incident to two or more edges in $Q$.
    \item $X$ contains no boundary vertices or edges.
  \end{compactenum}
  Then for any pair of boundary vertices $s$ and $t$, $G_{10}$ contains a path $P$ that 
  \begin{compactenum}
    \item begins at $s$ and ends at $t$
    \item contains all elements of $X$;
    \item goes straight through each vertex in $I$; and
    \item turns at each vertex in $L$.\todo{Define turns and goes straight}
    % \item contains no boundary vertices or edges except $s$ and $t$.
  \end{compactenum}
\end{lem}

\begin{proof}
  I have no idea how to prove this.  Computer search?.  
\end{proof}

A \emph{grid square} $S_{x,y,\ell}$ of side-length $\ell$ is a point set of the form $\{x,\ldots,x+\ell-1\}\times\{y,\ldots,y+\ell-1\}$ for some $x,y\in\Z^2$.  The \emph{interior} of $S_{x,y,\ell}$ is the grid square $S_{x+1,y+1,\ell-2}$

\begin{lem}\label{four_points}
  Let $P$ be a set of at most $4$ points in $\Z^2$.  Then there exists a set $S$ of disjoint grid squares, each of side-length $10$ such that each point of $P$ is contained in the interior of some square in $S$.
\end{lem}

\begin{proof}
  Let $P:=\{p_1,\ldots,p_t\}$ where $p_i:=(x_i,y_i)$.  Let $X:=\bigcup_{i=1}^t \{x_i\bmod 10,(x_{i+1})\bmod 10\}$ and let $Y:=\bigcup_{i=1}^t\{y_i\bmod 10,(y_i+1)\bmod 10\}$.  Then $|X|\le 2t\le 8$ and $|Y|\le 2t\le 8$.  In particular, there exists some $x\in\{0,\ldots,9\}\setminus X$ and some $y\in\{0,\ldots,9\}\setminus Y$.  We use squares of the form $\{x+10r,\ldots,x+10r-1\}\times\{y+10r\ldots,y+10r-1\}$ for each $r\in\Z$.
\end{proof}

\begin{lem}
  Let $P$ be a set of points in $\Z^2$ such that no square of side-length $80$ contains more than 4 points of $P$.  Then there exists a set $S$ of disjoint grid squares, each of side length $10$ and such that each point of $P$ is in the interior of some square in $S$.
\end{lem}

\begin{proof}
  Define a graph $H$ with vertex set $V(H):=P$ and that contains an edge $pq$ if and only if $\|p-q\|_\infty \le 20$. Observe that each component of $H$ contains at most $4$ points of $P$.  For each component $C$ of $H$, apply \cref{four_points} to $V(C)$.  This produces a set of disjoint grid squares of side length $10$ that each contain some point of $V(C)$ in their interior.  Any two squares $s_1$ and $s_2$ obtained from different components of $H$ are disjoint because $s_1$ contains a point $p_1\in P$ and $s_2$ contains a point $p_2\in P$ such that $\|p_1-p_2\|\infty > 20$, but the squares $s_1$ and $s_2$ each have side length $10$.\todo{Could be tidier here.}
\end{proof}

\todo[inline]{Adjust the numbers in the point covering lemmas.}

\begin{lem}
  Let $k$ be a positive integer, let $I,L$ be disjoint subsets of $V(G_{k})$, let $Q$ be a subset of $E(G_{k})$ and let $X:=I\cup L\cup Q$ have the following properties:
  \begin{compactenum}[(Pr1)]\setcounter{enumi}{0}
    \item Each $80\times 80$ subgrid of $G_k$ contains at most $4$ elements of $X$.
    \item $Q$ does not contain all the edges of a $4$-cycle.
    % \item $X$ contains no boundary vertices or edges.
    \item No vertex of $G_k$ is incident to three or more edges in $Q$.
    \item No vertex in $I\cup L$ is incident to two or more edges in $Q$.
  \end{compactenum}
  Then $G_k$ contains a path $P$ that
  \begin{compactenum}
    \item contains all elements of $X$;
    \item goes straight through each vertex in $I$; and
    \item turns at each vertex in $L$.
  \end{compactenum}
\end{lem}

\begin{proof}
  Start with the snakelike path that visits every $10$th row of $G_k$. Cover the elements of $X$ with disjoint squares of side length $10$.  Each such square is intersected by exactly one row of the snakelike path.  Use \cref{ten_by_ten} to locally deform the path to pick up the objects in any square that it intersects.
\end{proof}

A graph $H$ is an $a\times b$ \emph{pseudogrid} if there exists a set $\mathcal{P}$ of vertex-disjoint paths in $H$ such that the quotient graph $H':=H/\mathcal{P}$ is isomorphic to a subdivision of $G_{a\times b}$.  Since $G_{a\times b}$ has maximum degree $4$ any graph $G$ that contains a $G_{a\times b}$-minor contains a subgraph $H$ that is a $a\times b$ pseudogrid.\footnote{This comes from the fact that any tree $T$ with at most $4$ leaves contains at most two vertices of degree greater than $2$.  Contracting the path between these two vertices (if they exist) produces a tree with exactly one vertex of degree greater than $2$.}

\begin{lem}
  For any $k\times k$ pseudogrid $H$, $\chilin(G)\in\Omega(k)$.
\end{lem}

\begin{proof}
  This proof uses two large integer constants $\alpha$ and $\beta$ and two small positive real constants $\epsilon$ and $\delta$.  The relationships between these constants are that $\beta = \Theta(\alpha^{1/4})$ and $\epsilon=\Theta(1/\alpha)$.  Let $\varphi:V(H)\to\{1,\ldots,C\}$ be an arbitrarily $C$-colouring of some $k\times k$ pseudogrid $H$ for some $C\le \epsilon k$.  We will show that $H$ contains a path $P$ which has no centre with respect to $\varphi$.

  First we introduce some notations for mapping edges and vertices of the grid $G_k$ to their corresponding paths in the pseudogrid $H$.  For each vertex $v$ of $G_k$ there is a path $P_v$ in $\mathcal{P}$ (possibly consisting of one vertex) such that the vertex $v$ appears in $H'$ as a result of contracting $P_v$ in $H$.  In this way, each vertex $v$ of $G_k$ is associated with a non-empty \emph{colour set} $\phi(v):=\{\varphi(w):w\in V(P_v)\}$.  Similarly, for each edge $e$ of $G_k$ there is a path $P_e$ in $H'$ that comes from subdividing $e$ (possibly zero times).  Thus, each edge $e$ of $G_k$ is associated with a (possibly empty) \emph{colour set} $\phi(e):=\{\varphi(w):w\in P^-_e\}$, where $P^-_e$ denotes the (possibly empty) set of internal vertices on the path $P_e$.

  For each colour $c\in\{1,\ldots,C\}$, define the \emph{multiplicity} of $c$ as the number of vertices and edges of $G_k$ that include $c$ in their colour set.  Fix some large integer constant $\alpha$ and say that $\alpha$ is \emph{infrequent} if it has multiplicity less than $\alpha$ and \emph{frequent} otherwise.

  \begin{clm}\label{all_frequent}
    $H$ contains a $k'\times k'$ pseudogrid $H'$ in which every colour in $\{\varphi(v):v\in V(H')\}$ is frequent, for some $k'\ge (1-\epsilon \alpha)k$.
  \end{clm}

  \begin{proof}
    This proof uses the notion of \emph{removing} a row or column of a pseudogrid.  Removing a row is defined as follows (removing a column is similar): For each edge $e$ of $G_k$ in the relevant row, remove all edges of $P_e$ from $H$. Next, remove any isolated vertices from $H$ and finally, repeatedly remove degree-$1$ vertices from $H$ until none remain.  The result is pseudogrid with one less row.

    While there exists some infrequent colour $c\in\{1,\ldots,C\}$, find a row or column of $G_k$ that contains an edge or vertex with $c$ in its colour set and remove that row or column.  Doing this exhaustively results in the removal of at most $(\alpha-1) C \le \epsilon\alpha k$ rows and at most $\epsilon\alpha k$ columns of $H$.  This leaves a pseudogrid $H'$ that contains a $k'\times k'$ pseudogrid for some $k'\ge (1-\epsilon \alpha)k$ for which every colour that appears colours at least one vertex of $H'$ is frequent.
  \end{proof}

  Because of \cref{all_frequent}, we now assume that there is no infrequent colour, i.e., each colour $c\in\{1,\ldots,C\}$ has multiplicity at least $\alpha$.  Next we run a two stage process that, for each $c\in\{1,\ldots,C\}$, will identify a pair $(x_c,y_c)$ where each of $x_c$ and $y_c$ is an edge or vertex of $G_k$ that contains $c$ in its colour set.  These pairs are carefully chosen so that we will be able to construct a simple path $P_k$ in $G_k$ that contains $\bigcup_{c\in\{1,\ldots,C\}} \{x_c,y_c\}$ and has some additional nice properties.  Using these additional properties, we will then show how to convert $P_k$ into a simple path $P$ in $H$ that contains at least two vertices of each colour.

  \begin{compactenum}[{Stage} 1:]
    \item We begin this stage by defining each vertex and edge of $G_k$ as being \emph{unmarked}.  Fix some integer constant $\beta>0$.  We iterate over each $c\in\{1,\ldots,C\}$ in turn.  If $G_k$ contain a pair $(x_c,y_c)$ where each of $x_c$ and $y_c$ is an unmarked vertex or unmarked edge of $G_k$ with $c$ in its colour set and the distance in $G_k$ between $x_c$ and $y_c$ is at least $B$ then Stage~1 \emph{succeeds} for $c$.\todo{Define distance, including between sets}  We then \emph{mark} each vertex and edge $z$ of $G_k$ such that $\dist_{G_k}(z, \{x,y\})\le \beta$.

    If $G_k$ contains no such pair $(x_c,y_c)$, then we leave $x_c$ and $y_c$ undefined for now and say that Stage~1 \emph{fails} for colour $c$.  Note that Stage~1 can only fail for $c$ if $G_k$ contains fewer than $3(\beta/2+1)^2$\todo{check precisely} unmarked vertices/edges with $c$ in their colour sets.\footnote{$G_{a\times a}$ contains fewer than $3a^2$ vertices and edges and contains pairs of vertices/edges at distance $2(a-1)$.}  In particular, for $\alpha > 3(\beta/2+1)^2$, Stage~1 succeeds for $c=1$.

    \item Let $S\subset\{1,\ldots,C\}$ be the set of colours for which Stage~1 succeeds and let $\overline{S}:=\{1,\ldots,C\}\setminus S$ be the set of colours for which Stage~1 fails.  Let $R_1:=\bigcup_{c\in S}\{x_c,y_c\}$ be the set of vertices and edges chosen in $R_1$.
    
    As mentioned above, for sufficiently large $\alpha$ (relative to $\beta$), the set $S$ is non-empty and therefore $R_1$ is non-empty. In the next paragraph, we will define a function $\nu$ that maps each marked vertex or edge of $G_k$ to some element of $R_1$.  We use to this to construct a bipartite graph $I$ with vertex parts $A:=\overline{S}$ and $R_1$. The edge set of $I$ contains an edge $cx$ precisely when there is a marked edge or vertex $z$ of $G_k$ such that $c\in\phi(z)$ and $\nu(z)=x$.

    We now define the mapping $\nu$.  Fix some total order on $R_1$ by assigning each element of $R_1$ a distinct rank in $\{1,\ldots,|R_1|\}$.  For each marked element $z\in V(G_k)\cup E(G_k)$, let $\gamma(z):=\min\{\dist_{G_k}(z,x):x\in R_1\}$, let $\Gamma(z):=\{x\in R_1:\dist_{G_k}(z,x) \le (1+\delta)\gamma(z)$, and let $\nu(z)$ be the element in $\Gamma(z)$ of minimum rank.  In words, $\gamma(z)$ is the distance to $z$'s nearest neighbour in $R_1$, $\Gamma(z)$ is the set of all approximate nearest neighbours of $z$ in $R_1$ and $\nu(z)$ is the approximate nearest-neighbour of $z$ with smallest rank.  
    
    Note that for any marked element $z$, there exists $x\in R_1$ such that $\dist_{G_k}(z,x)\le\beta$.  Therefore $\gamma(z)\le\beta$.  Therefore, for any $x\in R_1$, the set $\nu{^-1}(x)$ only contains marked elements $z$ for which $\dist_{G_k}(x,z)\le (1+\delta)\beta$  Thus, $|\nu^{-1}(x)|\le \zeta\beta^2$ for some $\zeta$ depending only on $\delta$. This has two implications:
    \begin{compactenum}
      \item For each $x\in R_1$,  $\deg_{I}(x) \le \zeta\beta^2$.
      \item Since each colour $c\in A$ is frequent, $\deg_I(c)\ge \alpha/\zeta beta^2$.  
    \end{compactenum}
    These two conditions imply that, for each $A_0\subseteq A$, $|N_I(A_0)|\ge \alpha/\zeta^2\beta^4$.   In particular, for any $\alpha$ and $\beta$ with $\alpha > 2\zeta^2\beta^4$,  $|N_I(A_0)|\ge 2|A_0|$.  Therefore, by \cref{hall_vees}, there exists a subgraph $\Lambda$ of $I$ in which each vertex of $A$ has degree $2$ and each vertex of $B$ has degree at most one.  Therefore, each $c\in A=\overline{S}$ has two $\Lambda$-neighbours $p$ and $q$ in $B$.  The edge $cp$ is present in $I$ because $G_k$ contains a vertex $x_c$ with $s(x_c)=p$ and $c\in \phi(x_c)$.  Similarly, the edge $cq$ is present in $I$ because $G_k$ contains a vertex $y_c$ with $s(y_c)=q$ and $c\in\phi(y_c)$.  In this way, the graph $\Lambda$ defines the pairs $\{x_c,y_c\}$ for each $c\in\overline{S}$.
  \end{compactenum}

  Summarizing our work thus far, we have identified a subset  $R:=\bigcup_{c=1}^C\{x_c,y_c\}$ of at most $2c$ vertices and edges of $G_k$ where $c\in \phi(x_c)\cap \phi(y_c)$ for each $c\in\{1,\ldots,C\}$. Ultimately, we want to use these to find a path in $H$ that contains at least two vertices of each colour.  We start by first establishing some properties of $R\subseteq V(G_k)$.

  Let $R_1:=z\in\bigcup_{c\in S}\{x_c,y_c\}$ and $R_2:=z\in\bigcup_{c\in \overline{S}}\{x_c,y_c\}$ be the subsets of $R$ obtained during Stages~1 and 2, respectively. The following claim follows immediately from the algorithm used to select pairs in Stage~1.
  \begin{clm}\label{r1_far}
      For each distinct pair $z,z'\in R_1$, $\dist_{G_k}(z,z')\ge \beta$.
  \end{clm}
  
  The next claim states that any element $z'\in R_2$ is quite far from every element of $R_1$ except possibly its nearest neighbour in $R_1$.
  
  \begin{clm}\label{far_from_home}
    For each $x\in R_1$ and $y\in R_2$ with $\nu(y)\neq x$, $\dist_{G_k}(x,y)\ge \beta /(2+\delta)$.
  \end{clm}

  \begin{proof}
    Since $\nu(y)\neq x$, there exists some other $z\in R_1\setminus\{y\}$ with $\nu(y)=\nu(z)$.  Since $x,z\in R_1$, \cref{r1_far} implies that $\dist_{G_k}(x,z)\ge \beta$.  Since $\nu(y)=z$, $\dist_{G_k}(y, z) \le (1+\delta)\gamma(y)\le (1+\delta)\dist_{G_k}(y, x)$.  By the triangle inequality
    \[ \beta \le \dist_{G_k}(x,z) 
        \le \dist_{G_k}(x,y)+\dist_{G_k}(y,z)
        % \le \dist_{G_k}(x,y) + (1+\delta)\dist_{G_k}(x,y) 
        \le (2+\delta)\dist_{G_k}(x,y) \enspace . \qedhere
    \]
  \end{proof}
  
  
  \begin{clm}
    Any subset of $R$ containing containing $4$ elements contains a pair $(x,y)$ such that $\dist_{G_k}(x,y)\ge \psi\beta$, for some $\psi>0$ that depends only on $\delta$.
  \end{clm}
  
  \begin{proof}
    Let $R':=\{x_1,x_2,x_3,x_4\}$ be a $4$-element subset of $R$.  If $|R'\cap R_1|\ge 2$ because say $x_1,x_2\in R_1$ then $\dist_{G_k}(x_1,x_2)\ge \beta$, by \cref{r1_far}.  If $|R'\cap R_1=1$ because say $x_1\in R_1$ then $\nu(x_i)= x_1$ for at most one $i\in\{2,3,4\}$. For each of the two $j\in\{2,3,4\}\setminus\{i\}$, $\dist_{G_k}(x_1,x_j)\ge \beta/(2+\delta)$, by \cref{far_from_home}.
    
    The only possibility that remains is that $R'\subseteq R_2$. Let the elements of $R'$ be labelled so that the rank of $\nu(x_i)$ is less than the rank of $\nu(x_j)$ for each $i<j$. Suppose that $\dist_{G_k}(x_i,x_j)\le \psi\beta$ for each $i,j\in\{1,2,3\}$, otherwise there is nothing to prove.
    
    First observe that, for any $i\in\{1,2,3,4\}$,  $\dist_{G_k}(x_i,\nu(x_i)) \ge (\tfrac{1}{2+\delta}-\psi)\beta$ since, otherwise 
    \[  \dist_{G_k}(x_{i+1},\nu(x_i))\le \dist_{G_k}(x_{i+1},x_i) + 
    \dist_{G_k}(x_i,\nu(x_i)) < \beta/(2+\delta) \enspace ,
    \]
    for any $j\in\{1,2,3,4\}\setminus\{i\}$, 
    contradicting \cref{far_from_home}.\todo{Make this a separate lemma?}
    
  \end{proof}

  \begin{clm}
      There is no vertex of $G_k$ that has three or more incident edges in $R$.
  \end{clm}

  \begin{proof}
    To prove this, we have to define $s(x)$ a little more carefully.
  \end{proof}
  
  \begin{clm}
    There is no triple $(v,e_1,e_2)\in R$ where $v$ is a vertex of $G_k$ and $e_1$ and $e_2$ are each edges of $G_k$ incident on $v$.
  \end{clm}

  \todo[inline]{The following claim is not true.  It's easy to come up with an example in which a small square contains $8$ different elements.  We're probably better off adjusting the boundaries of Voronoi cells so that we enforce a minimum distance (say at least $\beta/10$) between Voronoi vertices.}
  \begin{clm}
    Any $5$-element subset of $R$ contains a pair $x,y$ with $\dist_{G_k}(x,y)\ge\beta/2$.
  \end{clm}

  \begin{proof}
    Let $R'$ be some $5$-element subset of $R$. $R'$ contains two or more elements of $R_1$ then the distance between these two elements is at most $\beta>\beta/2$.  Therefore, we may assume that $R'$ contains at most one element in $R_1$.  If $R'$ contains exactly one element $x$ in $R_1$ then, by \cref{far_from_home}, there is at most one other element in $R'$ whose distance to $x$ is at most $\beta/2$, and the three remaining elements have greater distance.  Therefore, $R'$ contains no elements in $R_1$.  Therefore, $R'':=\{s(x):x\in R'\}$ is a five-element set whose pairwise distances are all at most $5\beta/2$ and whose minimum distance is at least $\beta$ (Contradiction?)\todo{Finish}.
  \end{proof}

  \begin{cor}
      In any $\beta/2\times \beta/2$ subgrid $G_k$ contains at most $4$ elements of $R$.
  \end{cor}

    Now classify each vertex of $G_k$ as either being \emph{straight} or \emph{bent}. Define what it means for a path $P$ to go straight or bend at a vertex.  Now we should be able to finish with this lemma:

    \begin{lem}
        $G_k$ contain a simple path $P$ that contains every edge and vertex in $R$.  Furthermore, $P$ bends at each bent vertex in $R$ and $P$ goes straight at each straight vertex in $R$.
    \end{lem}
\end{proof}







\bibliographystyle{plainurlnat}
\bibliography{lin-vs-cen}




\end{document}
